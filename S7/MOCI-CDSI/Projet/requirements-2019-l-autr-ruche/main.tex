%Copyright 2014 Jean-Philippe Eisenbarth
%This program is free software: you can 
%redistribute it and/or modify it under the terms of the GNU General Public 
%License as published by the Free Software Foundation, either version 3 of the 
%License, or (at your option) any later version.
%This program is distributed in the hope that it will be useful,but WITHOUT ANY 
%WARRANTY; without even the implied warranty of MERCHANTABILITY or FITNESS FOR A 
%PARTICULAR PURPOSE. See the GNU General Public License for more details.
%You should have received a copy of the GNU General Public License along with 
%this program.  If not, see <http://www.gnu.org/licenses/>.

%Based on the code of Yiannis Lazarides
%http://tex.stackexchange.com/questions/42602/software-requirements-specification-with-latex
%http://tex.stackexchange.com/users/963/yiannis-lazarides
%Also based on the template of Karl E. Wiegers
%http://www.se.rit.edu/~emad/teaching/slides/srs_template_sep14.pdf
%http://karlwiegers.com
\documentclass{scrreprt}
\usepackage{listings}
\usepackage{underscore}
\usepackage[bookmarks=true]{hyperref}
\usepackage[utf8]{inputenc}
\usepackage[toc,acronym,section=section]{glossaries}
\usepackage[french]{babel}
\usepackage{graphicx}

% to suppress glossaries page break

\hypersetup{
    bookmarks=false,    % show bookmarks bar?
    pdftitle={Cahier des Charges},    % title
    pdfauthor={Jean-Philippe Eisenbarth},                     % author
    pdfsubject={TeX and LaTeX},                        % subject of the document
    pdfkeywords={TeX, LaTeX, graphics, images}, % list of keywords
    colorlinks=true,       % false: boxed links; true: colored links
    linkcolor=blue,       % color of internal links
    citecolor=black,       % color of links to bibliography
    filecolor=black,        % color of file links
    urlcolor=purple,        % color of external links
    linktoc=page            % only page is linked
}%
\def\myversion{0.1 }
\date{}
%\title
\usepackage{hyperref}
\newcounter{reqcounter}
\newcounter{reqfcounter}
\newtheorem{req}{Besoin}

\newenvironment{reqs}[1]{
\refstepcounter{reqcounter}
\textbf{Requirement \thereqcounter} \space #1\\
\begin{em}}{\end{em}\vspace{1em}}

\newenvironment{reqf}[1]{
\refstepcounter{reqfcounter}
\textbf{Requirement \thereqfcounter} \space #1\\
\begin{enumerate}}{\end{enumerate}\vspace{1em}}

\makeglossaries

%%%% DÉBUT RAPPORT %%%%

%%%% GLOSSAIRE %%%%

\newglossaryentry{latex}
{
        name=latex,
        description={Is a mark up language specially suited for 
scientific documents}
}
\newglossaryentry{livret}
{
        name={Livret de l'élève},
        description={Il contient toutes les informations concernant la scolarité des étudiants}
}


%%%% DÉBUT DOCUMENT %%%%
\begin{document}
\renewcommand*{\glsclearpage}{}
\begin{flushright}
    \rule{16cm}{5pt}\vskip1cm
    \begin{bfseries}
        \Huge{Cahier des charges}\\
        \vspace{1.9cm}
        Le projet de MOCI\\
        \vspace{1.9cm}
        \LARGE{Version \myversion draft}\\
        \vspace{1.9cm}
        Préparé par $<$\author$>$.\\
        \vspace{1.9cm}
        $<$Organization$>$\\
        \vspace{1.9cm}
        \today\\
    \end{bfseries}
\end{flushright}

\tableofcontents
\chapter*{Historique}

\begin{center}
    \begin{tabular}{|c|c|c|c|}
        \hline
	    Name & Date & Reason for changes & Version\\
        \hline
	    Francois Charoy & 29/8/2019 & initial & 0.1\\
        \hline
	    Francois Charoy & 29/6/2017 & initial & 0.1\\
        \hline
	    Martine Gautier & 30/6/2017 & orthographe & 0.1.1\\
        \hline
        Brigitte Wrobel-Dautcourt & 22/8/2017 & ponctuation, typo, orthographe, grammaire... & 0.1.2\\
        \hline
        Martine Gautier & 07/09/2017 & ponctuation, typo, orthographe, grammaire, ... & 0.1.3\\
        \hline
    \end{tabular}
\end{center}
\newpage
\chapter{Introduction}
Ce cahier des charges est réalisé dans le cadre du cours de TELECOM Nancy  et fait référence aux livres suivants~\cite{Sommerville:2010:SE:1841764,Pohl:2010:REF:1869735,Rumbaugh:2004:UML:993859}.
\section{Objectif}
TIMOTHEE\\\\\\\\\\\\\\\\\\
$<$Identifiez le produit dont le cahier des charges va être décrit dans ce document. Indiquez ce que couvre le produit, en particulier si le cahier des charges ne concerne qu'une partie d'un système.$>$

\section{Conventions}
TIMOTHEE\\\\\\\\\\\\\\\\
$<$Décrivez les conventions typographiques et les standards utilisés s'il vous en avez définie$>$

\section{Description du projet}
TIMOTHEE\\\\\\\\\\\\\\
$<$Donnez une description rapide du logiciel spécifié, son but, les objectifs métiers, les gains attendus.$>$


\section{Contexte et origine}
TIMOTHEE\\\\\\\\\\\\
$<$Décrivez le contexte dans lequel va fonctionner le logiciel et éventuellement ce qu'il va remplacer ou compléter. Décrivez également rapidement les relations avec les autres systèmes de l'environnement.$>$
\newacronym{ade}{ADE}{ADE est le logiciel d'emploi du temps de l'université de Lorraine}
\newacronym{API}{API}{Application Programming Interface}

L'emploi du temps des étudiants est accessible grâce aux \acrshort{API} d'\acrshort{ade}.

Les informations concernant les règles d'absence se trouvent dans le \Gls{livret}.

\section{Principales fonctionnalités}
TIMOTHEE\\
$<$Indiquez ici les fonctionnalités principales du produit ainsi que les acteurs et leurs caractéristiques. $>$

\section{Les acteurs}
TIMOTHEE\\
$<$Décrivez les différents acteurs concernés par le système aussi bien pour l'utilisation que pour l'exploitation.$>$

\section{Environnement opérationnel}
MARINE\\
$<$Décrivez le contexte dans lequel le logiciel va s'exécuter, où se trouvent les serveurs, qui va les opérer.$>$
L'application doit être hébergée sur un serveur web CentOS, couplé à un serveur de base de données qui permet d'enregistrer les informations.

\section{Contraintes d'implantation et de conception}
%QUENTIN\\

Ce projet étant un site web, le logiciel sera développé dans un langage orienté Web. Dans ce cadre, la solution retenue est Angular pour la partie frontend et PHP pour la partie backend. Nous utiliserons PHP en association avec Symfony pour simplifier le développement. Cette partie backend sera associée à une base de donnée. Une base de données en PostgreSQL semble plus adapté qu'une simple base de donnée en MySQL, vis-a-vis de la quantité de données à traiter.

L'ensemble du site sera mis en production sur un serveur dédié, hebergé chez OVH. Celui-ci tournera sous CentOS. Pour plus de sécurité, il devra contenir deux disques en RAID 1 pour garantir un niveau de service optimal d'une capacité de 100Go minimum chacun, ainsi qu'un processeur deux coeurs de 2 GHz et 4Go de RAM . Il récupéra l'ensemble des requêtes qu'il enverra au serveur NGINX. Il servira aussi de serveur de base de données. Elle sera exportée sur une machine du Gestionnaire pour limiter les pertes de données en cas de pannes majeur.

%%$<$Décrivez ce qui peut avoir un impact sur la mise en {\oe}uvre, comme des questions de réglementation, de support d'exécution, de limites techniques, d'outils à utiliser, de langage, de système.$>$

%%langages de prog, type de bdd, framework...

\section{Hypothèses et dépendances}
MARINE\\
$<$Quelles sont les dépendances et les hypothèses qui vont orienter la construction du logiciel ?$>$

Imaginer un certain nombre d'utilisateurs, la capacité du logiciel
les dépendances à des systèmes externes (cartes bleues, système d'authentification particulier ...)
On parlait d'un nombre de potentielles de connexions de 100 000 personnes.

\newpage
\chapter{Besoins utilisateurs}

\section{Besoins fonctionnels}
CHACUN REPREND CEUX DE SA PARTIE\\
$<$Décrivez sous forme de liste les besoins fonctionnels de l'application. Chaque besoin doit être formaté avec le modèle indiqué et surtout être numéroté.$>$


\begin{reqs}{Le système doit permettre d'éditer la liste des absences pour un étudiant.}
Le système doit permettre de retrouver toutes les absences d'un étudiant et de présenter une liste incluant les absences justifiées, le motif et les absences non justifiées.
\end{reqs}

\begin{reqs}{Le système doit permettre à un enseignant de saisir les absences du groupe d'étudiants avec qui il a cours.}
Le système doit permettre de retrouver toutes les absences d'un étudiant et de présenter une liste incluant les absences justifiées, le motif et les absences non justifiées.
\end{reqs}

%%% BESOINS RESPONSABLE PC

\begin{reqs}{Le système doit permettre à un responsable de point de collecte de s’inscrire au site.}
\end{reqs}

\begin{reqs}{Le système doit permettre à un responsable de point de collecte de proposer la création d’un nouveau point de collecte.}
\end{reqs}

\begin{reqs}{Le système doit permettre à un responsable de point de collecte d’accepter la participation d’un producteur dans un point de collecte dont il est responsable.}
\end{reqs}

\begin{reqs}{Le système doit permettre à un responsable de point de collecte de refuser la participation d’un producteur dans un point de collecte dont il est responsable.}
\end{reqs}

\begin{reqs}{Le système doit permettre à un responsable de point de collecte d’arrêter sa collaboration avec un producteur.}
\end{reqs}

\begin{reqs}{Le système doit permettre à un responsable de point de collecte de définir le jour et l’heure de remise des paniers.}
\end{reqs}

\begin{reqs}{Le système doit permettre à un responsable de point de collecte d’ouvrir les ventes sur un point de collecte.}
Les producteurs pourront ainsi proposer leurs produits et les consommateurs pourront les commander.
\end{reqs}

\begin{reqs}{Le système doit permettre à un responsable de point de collecte de fermer les ventes sur un point de collecte.}
Les ventes seront closes et les consommateurs ne pourront plus commander jusqu’au prochain cycle.
\end{reqs}

\begin{reqs}{Le système doit permettre à un responsable de point de collecte de pointer les livraisons.}
Le responsable pourra ainsi noter tout ce qui a été livré et ce qui ne l’a pas été.
\end{reqs}

\begin{reqs}{Le système doit permettre à un responsable de point de collecte de reporter les manquements ou les erreurs sur les commandes.}
Le responsable pourra ainsi faire remonter les oublis dans les paniers mais aussi les produits en mauvais états.
\end{reqs}

\begin{reqs}{Le système doit permettre à un responsable de point de collecte de mettre en avant certains articles.}
Les articles mis en avant seront remontés dans la liste de tous les articles du point de collecte et mis en avant avec un style différent.
\end{reqs}

\begin{reqs}{Le système doit permettre à un responsable de point de collecte de définir les cycles de vente d’un point de collecte.}
Le responsable pourra décider s’il fait des cycles de X jours de commande avant réception de la marchandise.
\end{reqs}

\begin{reqs}{Le système doit permettre à un responsable de point de collecte de demander la fermeture définitive d’un point de collecte.}
\end{reqs}

\begin{reqs}{Le système doit permettre à un responsable de point de collecte de définir une date limite de commande.}
\end{reqs}

\begin{reqs}{Le système doit permettre à un responsable de point de collecte de se désinscrire du site.}
\end{reqs}

\begin{reqs}{Le système doit permettre à un responsable de point de collecte de fermer temporairement un point de collecte.}
\end{reqs}

\begin{reqs}{Le système doit permettre à un responsable de point de collecte de rouvrir un point de collecte qu’il a fermé.}
\end{reqs}

\begin{reqs}{Le système doit permettre à un responsable de point de collecte de changer de mot de passe en cas d’oubli.}
\end{reqs}

%%% FIN BESOINS RESPONSABLE PC

%%% BESOINS GESTIONNAIRE DE PLATEFORME

\begin{reqs} système doit permettre au gestionnaire de la plateforme de valider la création d'un point de collecte}
\end{reqs}

\begin{reqs}{Le système doit permettre au gestionnaire de la plateforme d'invalider la création d'un point de collecte}
\end{reqs}

\begin{reqs}{Le système doit permettre au gestionnaire de la plateforme de fermer un point de collecte}
\end{reqs}

\begin{reqs}{Le système doit permettre au gestionnaire de la plateforme de visualiser les informations d'un point de collecte}
\end{reqs}

\begin{reqs}{Le système doit permettre au gestionnaire de la plateforme de définir le pourcentage de CA que prend le responsable d'un point de collecte}
\end{reqs}

\begin{reqs}{Le système doit permettre au gestionnaire de la plateforme de définir le pourcentage de CA que prend le gestionnaire de la plateforme}
\end{reqs}

\begin{reqs}{Le système doit permettre au gestionnaire de la plateforme de rembourser les clients en fonction des erreurs remontées par responsable du point de collecte}
\end{reqs}

\begin{reqs}{Le système doit permettre au gestionnaire de la plateforme de consulter les remontés des responsables des points de collecte}
\end{reqs}

\begin{reqs}{Le système doit permettre au gestionnaire de la plateforme de contrôler un point de collecte}
\end{reqs}

\begin{reqs}{Le système doit permettre au gestionnaire de la plateforme de créer des comptes pour les employés du gestionnaire de la plateforme}
\end{reqs}

\begin{reqs}{Le système doit permettre au gestionnaire de la plateforme de récupérer le mot de passe des comptes}
\end{reqs}

%%% FIN BESOINS GESTIONNAIRE DE PLATEFORME


%%% BESOINS UTILISATEURS
\begin{reqs}{Le système doit permettre à un utilisateur de s'inscrire}
Le système doit permettre à toute personne de s'inscrire à la plateforme à partir du moment ou elle détient une adresse mail qui lui permettra d'être enregistrée.
\end{reqs}

\begin{reqs}{Le système doit permettre à un utilisateur de se désinscrire}
Le système doit permettre à toute personne inscrite à la plateforme, de se désinscrire facilement sans avoir à donner de justification.
\end{reqs}

\begin{reqs}{Le système doit permettre à un utilisateur de pouvoir demander la création d'un nouveau mot de passe}
Le système doit permettre à toute personne inscrite ayant oublié son mot de passe de pouvoir le modifier à partir d'un lien envoyé par mail. Le nouveau mot de passe doit être enregistré dans la base de données à la place de l'ancien.
\end{reqs}

\begin{reqs}{Le système doit permettre à un utilisateur de sélectionner un point de collecte}
Le système doit permettre à l'utilisateur de visualiser les points de collecte qui sont dans un rayon de 20km autour d'une localisation entrée au préalable, et ainsi sélectionner celui de son choix.
\end{reqs}

\begin{reqs}{Le système doit permettre à un utilisateur de pouvoir commander des produits}
Le système doit permettre à un utilisateur de visualiser les produits disponibles dans le point de collecte sélectionner et lui permettre de les mettre dans son panier avant de confirmer et payer la commande.
\end{reqs}

\begin{reqs}{Le système doit permettre à un utilisateur de pouvoir annuler une commande}
Le système doit permettre à l'utilisateur d'annuler sa commande durant la durée qui lui sera renseignée s'il change d'avis. +++ VOIR SI ON ANNULE TOUTE UNE COMMANDE OU JUSTE UN PRODUIT PARTICULIER
\end{reqs}

\begin{reqs}{Le système doit permettre à un utilisateur de visualiser les informations concernant les produits}
Le système doit permettre d'avoir accès à toutes les informations sur les produits, à savoir les prix, quantités et provenances.
\end{reqs}

\begin{reqs}{Le système doit permettre à un utilisateur de visualiser le contenu de son panier}
Le système doit permettre à l'utilisateur de visualiser son panier contenant tous les produits qu'il a sélectionné. Il doit permettre également à l'utilisateur de supprimer des produits ou encore d'ajouter plus de quantités aux produits présents dans le panier.
\end{reqs}

\begin{reqs}{Le système doit permettre à un utilisateur de payer sa commande en ligne}
Le système doit permettre à l'utilisateur de payer sa commande en ligne en dirigeant ce dernier vers un service de paiement en ligne sécurisé.
\end{reqs}

\begin{reqs}{Le système doit permettre à un utilisateur le remboursement de sa commande}
Le système doit permettre à l'utilisateur qui a annulé sa commande de pouvoir être remboursé à la fin des délais de commande.
\end{reqs}

\begin{reqs}{Le système doit permettre à un utilisateur de signaler un problème}
Le système doit permettre à l'utilisateur de signaler un problème sur la plateforme en ligne, notamment concernant l'absence de produits à la collecte, une qualité médiocre ou encore concernant les points de collecte en général ++ VOIR POUR D'AUTRES
\end{reqs}


%%% FIN BESOIN UTILISATEURS


\section{Besoins non fonctionnels}
$<$Décrivez sous forme de liste les besoins non fonctionnels de l'application. Chaque besoin doit être formaté avec le modèle indiqué et surtout être numéroté. Pour chaque besoin non fonctionnel, vous devez indiquer comment il sera vérifié.$>$

\subsection{Performance}
SARAH\\
\begin{reqs}{Le temps de réponse pour les actions de l'utilisateur doit être inférieur à 100ms dans 99\% des cas.}
Le système doit être réactif et ne pas ralentir le travail de l'utilisateur. Des tests de performance seront mis en place pour vérifier ce besoin ainsi qu'une solution de monitoring pour vérifier que la performance attendue est bien atteinte.
\end{reqs}

\subsection{Sureté}
%QUENTIN\\
\begin{reqs}{Le système doit pouvoir être mis à jour sans être interrompu}
Les mises à jour logicielles doivent pouvoir être effectuées sans interruption du service.
\end{reqs}

\begin{reqs}{Le système doit être protégé contre les attaques DDoS}
Le système doit être capable de résister à une attaque par dénie de service (DDoS). Un simulation avec une société de sécurité pourra être envisagé afin de tester la protection.
\end{reqs}

\begin{reqs}{Le système doit être sauvegardé sur des supports multiples}
Les sauvegardes devrons être réalisés sur des supports multiples. Y compris sur des supports non relié au réseau en permanence pour prévenir en cas d'attaque sur le réseau informatique de l'entreprise. Que celle-ci proviennent de l'intérieur ou de l'extérieur de cette dernière.
\end{reqs}

\begin{reqs}{Le système doit envoyer une alerte en cas de connexion trop nombreuses}
Le système doit envoyer une alerte si de trop nombreuses connexions sont détectées. Elles permettront d'anticiper les éventuels problèmes liés à une attaque.
\end{reqs}

\begin{reqs}{Le système doit être sauvegardé de manière régulière}
Le système doit être sauvegardé une fois par jour, de préférence, la nuit entre 2h et 3h du matin.
\end{reqs}

\begin{reqs}{Le système doit déconnecté les utilisateurs au bout d'un certain moment d'inactivité}
Le système doit déconnecté les utilisateurs au bout d'un certain moment d'inactivité.
\end{reqs}

\begin{reqs}{Le système doit avoir une interface de paiement sécurisée}
Le système doit avoir une interface de paiement sécurisée. Celle-ci disposera d'une connexion forte 
\end{reqs}

\subsection{Sécurité}
TIMOTHEE\\
\begin{reqs}{Le système doit fournir un moyen d'authentification à 2 facteurs}
\end{reqs}

\subsection{Autres qualités}

\begin{reqs}{Le système doit pouvoir supporter 210 000 connexions simultanées.}
Le système doit être réactif et ne pas ralentir le travail de l'utilisateur. Des tests de capacité seront mis en place pour vérifier ce besoin.
\end{reqs}

\begin{reqs}{Le système doit pouvoir sauvegarder toutes les données de l'application.}
Cela fera preuve d'une maintenance et d'une attention accrue.
\end{reqs}


\section{Besoins liés au domaine}
RGPD MARINE\\ (détailler les fonctions qu'il faudrait mettre en oeuvre (ex: bouton pour récupérer toutes les données))
Tout ce qui sont des contraintes venant de l'extérieur de l'application (loi, règlement...) qui imposent des contraintes
(ex: on peut pas conserver les cryptogrammes d'une carte bleue )

\begin{reqs}{La système doit répondre aux contraintes réglementaires liées à la RGPD}
\item 
\end{reqs}
\newpage
\chapter{Besoins système}

\section{Besoins système fonctionnels}
CHACUN REPREND CEUX DE SA PARTIE
$<$Pour chaque besoin fonctionnel, donnez une description détaillée de son fonctionnement permettant de comprendre plus précisément son fonctionnement. La numérotation doit correspondre aux besoins utilisateurs.$>$


\begin{reqf}{Le système doit permettre d'éditer la liste des absences pour un étudiant.}
\item L'utilisateur peut rechercher un étudiant à partir de son nom ou de son numéro INE. 
\item La recherche est facilitée par un mécanisme d'auto-complétion. 
\item La liste des absences s'affiche à l'écran : date, motif, justification.
\item Le total des absences justifiées et non justifiées est également affiché.
\end{reqf}

%%% BESOINS RESPONSABLE PC

\begin{reqf}{Le système doit permettre d'éditer la liste des absences pour un étudiant.}
\item L'utilisateur peut rechercher un étudiant à partir de son nom ou de son numéro INE. 
\item La recherche est facilitée par un mécanisme d'auto-complétion. 
\item La liste des absences s'affiche à l'écran : date, motif, justification.
\item Le total des absences justifiées et non justifiées est également affiché.
\end{reqf}

\begin{reqf}{Le système doit permettre à un responsable de point de collecte de s'inscrire au site.}
\item L'utilisateur a accès à une page de création de compte.
\end{reqf}

\begin{reqf}{Le système doit permettre à un responsable de point de collecte de proposer la création d'un nouveau point de collecte.}
\item Le responsable peut saisir les coordonnées du point de collecte qu'il veut créer.
\item Le responsable a accès à sa liste des points de collecte et un témoin lui indique son état (créé, en attente, création refusée).
\end{reqf}

\begin{reqf}{Le système doit permettre à un responsable de point de collecte d'accepter la participation d'un producteur dans un point de collecte dont il est responsable.}
\item En ouvrant la fiche d'un point de collecte, le responsable a accès à la liste des producteurs souhaitant faire parties du point de collecte.
\item Le clic sur un producteur ouvre une fenêtre lui permettant d'accepter la collaboration ou non.
\end{reqf}

\begin{reqf}{Le système doit permettre à un responsable de point de collecte de refuser la participation d'un producteur dans un point de collecte dont il est responsable.}
\item En ouvrant la fiche d'un point de collecte, le responsable a accès à la liste des producteurs souhaitant faire parties du point de collecte.
\item Le clic sur un producteur ouvre une fenêtre lui permettant d'accepter la collaboration ou non.
\end{reqf}

\begin{reqf}{Le système doit permettre à un responsable de point de collecte d'arrêter sa collaboration avec un producteur.}
\item En ouvrant la fiche d'un point de collecte, le responsable a accès à la liste des producteurs faisant parties du point de collecte.
\item Le clic sur un producteur ouvre une fenêtre lui permettant d'arrêter la collaboration ou non.
\end{reqf}

\begin{reqf}{Le système doit permettre à un responsable de point de collecte définir le jour et l'heure de remise des paniers.}
\item En ouvrant la fiche d'un point de collecte, le responsable a accès aux paramètres du point de collecte.
\item Dans les paramètres du point de collecte se trouvent le jour et la remise des paniers.
\end{reqf}

\begin{reqf}{Le système doit permettre à un responsable de point de collecte d'ouvrir les ventes sur un point de collecte.}
\item En ouvrant la fiche d'un point de collecte, le responsable peut ouvrir les ventes du point de collecte.
\end{reqf}

\begin{reqf}{Le système doit permettre à un responsable de point de collecte de fermer les ventes sur un point de collecte.}
\item En ouvrant la fiche d'un point de collecte, le responsable peut fermer les ventes du point de collecte.
\end{reqf}

\begin{reqf}{Le système doit permettre à un responsable de point de pointer les livraisons.}
\item Depuis la fiche d'un point de collecte, le responsable a accès à l'inventaire des produits commandés.
\item Cet inventaire lui permet de cocher la bonne réception des produits et d'indiquer les éventuelles remarques concernant ces derniers.
\end{reqf}

\begin{reqf}{Le système doit permettre à un responsable de point de collecte de reporter les manquements ou les erreurs.}
\item Depuis la fiche d'un point de collecte, le responsable a accès à l'inventaire des produits commandés.
\item Cet inventaire lui permet de cocher la bonne réception des produits et d'indiquer les éventuelles remarques concernant ces derniers.
\item Dans l'inventaire le responsable peut saisir les quantités manquantes ou en surplus.
\end{reqf}

\begin{reqf}{Le système doit permettre à un responsable de point de collecte de mettre en avant certains articles.}
\item Depuis la fiche d'un point de collecte, le responsable a accès à la liste de tous les articles mis en vente pendant le cycle.
\item Depuis la liste de tous les articles mis en vente pendant le cycle, le responsable peut à l'aide d'un bouton sélectionner des articles qui apparaîtront en tête de liste.
\end{reqf}

\begin{reqf}{Le système doit permettre à un responsable de point de collecte de définir les cycles de vente d'un point de collecte.}
\item Depuis la fiche d'un point de collecte, le responsable a accès aux paramètres du point de colecte.
\item Depuis les paramètres du point de collecte, le responsable peut définir le nombre de jours que dure un cycle de vente.
\end{reqf}

\begin{reqf}{Le système doit permettre à un responsable de point de collecte de demander la fermeture définitive d'un point de collecte.}
\item Depuis la fiche d'un point de collecte, le responsable a accès aux paramètres du point de colecte.
\item Depuis les paramètres du point de collecte, le responsable peut demander la fermeture définitive d'un point de collecte que le système lui demandera de confirmer.
\end{reqf}

\begin{reqf}{Le système doit permettre à un responsable de point de collecte de définir une date limite de commande.}
\item Depuis la fiche d'un point de collecte, le responsable a accès aux paramètres du point de colecte.
\item Depuis les paramètres du point de collecte, le responsable peut définir le nombre de jours avant lesquels il n'est plus possible de commander.
\end{reqf}

\begin{reqf}{Le système doit permettre à un responsable de point de collecte de se désinscrire du site.}
\item Depuis sa fiche utilisateur, le responsable peut demander sa désinscription au site, le système lui demandera une confirmation.
\item Si un responsable se désinscrit du site, tous les points de collecte dont il est responsable seront soit transférés à un autre responsable par le gestionnaire, soit fermé par le gestionnaire.
\end{reqf}

\begin{reqf}{Le système doit permettre à un responsable de point de collecte de fermer temporairement un point de collecte.}
\item Depuis la fiche d'un point de collecte, le responsable a accès aux paramètres du point de colecte.
\item Depuis les paramètres du point de collecte, le responsable peut demander la fermeture temporaire du point de collecte que le système lui demandera de confirmer.
\end{reqf}

\begin{reqf}{Le système doit permettre à un responsable de point de collecte de rouvrir un point de collecte qu'il a fermé.}
\item Depuis la fiche d'un point de collecte, le responsable a accès aux paramètres du point de colecte.
\item Depuis les paramètres du point de collecte, le responsable peut demander la ré-ouverture du point de collecte que le système lui demandera de confirmer.
\end{reqf}

\begin{reqf}{Le système doit permettre à un responsable de point de collecte de changer de mot de passe en cas d'oubli.}
\item Depuis l'écran de connexion, l'utilisateur aura accès à une option "Mot de passe oublié".
\item Cette option lui demandera de saisir son adresse email, et enverra le mot de passe temporaire à l'utilisateur.
\item Lors de la connexion d'un utilisateur avec un mot de passe temporaire, le système force l'utilisateur a changer de mot de passe.
\end{reqf}


%%% FIN BESOINS RESPONSABLE PC



%%% Besoins utilisateurs Gestionnaire de plateforme

\begin{reqf}{Le système doit permettre au gestionnaire de la plateforme de valider la création d'un point de collecte}
\item Le gestionnaire de la plateforme doit pouvoir voir les demandes de création de points de collecte
\item Le gestionnaire de la plateforme doit pouvoir valider les demandes de création de points de collecte
\end{reqf}

\begin{reqf}{Le système doit permettre au gestionnaire de la plateforme d'invalider la création d'un point de collecte}
\item Le gestionnaire de la plateforme doit pouvoir voir les demandes de création de points de collecte
\item Le gestionnaire de la plateforme doit pouvoir proposer des suggestions de modifications
\item Le gestionnaire de la plateforme doit pouvoir invalider les demandes de création de points de collecte
\end{reqf}

\begin{reqf}{Le système doit permettre au gestionnaire de la plateforme de fermer un point de collecte}
\item Le gestionnaire de la plateforme doit pouvoir fermer un point de collecte
\end{reqf}

\begin{reqf}{Le système doit permettre au gestionnaire de la plateforme de visualiser les informations d'un point de collecte}
\item Le gestionnaire de la plateforme doit pouvoir voir le chiffre d'affaire du point de collecte
\item Le gestionnaire de la plateforme doit pouvoir voir sa fréquentation du point de collecte
\item Le gestionnaire de la plateforme doit pouvoir voir la localisation du point de collecte
\item Le gestionnaire de la plateforme doit pouvoir voir le taux de remboursement du point de collecte
\item Le gestionnaire de la plateforme doit pouvoir voir le pourcentage du chiffre d'affaire du point de collecte destiné au responsable du point de collecte
\item Le gestionnaire de la plateforme doit pouvoir voir le pourcentage du chiffre d'affaire du point de collecte destiné au gestionnaire de la plateforme
\end{reqf}

\begin{reqf}{Le système doit permettre au gestionnaire de la plateforme de définir le pourcentage de CA que prend le responsable d'un point de collecte}
\item Le gestionnaire de la plateforme doit pouvoir voir le pourcentage du chiffre d'affaire du point de collecte destiné au responsable du point de collecte
\item Le gestionnaire de la plateforme doit pouvoir définir le pourcentage du chiffre d'affaire du point de collecte destiné au responsable du point de collecte
\end{reqf}

\begin{reqf}{Le système doit permettre au gestionnaire de la plateforme de définir le pourcentage de CA que prend le gestionnaire de la plateforme}
\item Le gestionnaire de la plateforme doit pouvoir voir le pourcentage du chiffre d'affaire du point de collecte destiné au gestionnaire de la plateforme
\item Le gestionnaire de la plateforme doit pouvoir définir le pourcentage du chiffre d'affaire du point de collecte destiné au gestionnaire de la plateforme
\end{reqf}

\begin{reqf}{Le système doit permettre au gestionnaire de la plateforme de rembourser les clients en fonction des erreurs remontées par responsable du point de collecte}
\item Le système doit afficher au gestionnaire du point de collecte les erreurs remontées par les responsables de point de collecte
\item Le système doit permettre de déclencher le remboursement pour les produits en question
\end{reqf}

\begin{reqf}{Le système doit permettre au gestionnaire de la plateforme de consulter les remontés des responsables des points de collecte}
\item Le système doit afficher au gestionnaire du point de collecte les erreurs remontées par les responsables de point de collecte
\end{reqf}

\begin{reqf}{Le système doit permettre au gestionnaire de la plateforme de contrôler un point de collecte}
\item Le système doit permettre au gestionnaire de la plateforme d'accéder aux informations sur d'un point de collecte
\item Le système doit permettre d'ajouter des notes sur un point de collecte lors d'un controle
\item Le système doit permettre de programmer un nouveau controle si besoin -
\end{reqf}

\begin{reqf}{Le système doit permettre au gestionnaire de la plateforme de créer des comptes pour les employés du gestionnaire de la plateforme}
\item Le système doit permettre au gestionnaire de la plateforpme de créer un compte pour chacun de ses employés
\item Le système doit permettre au gestionnaire de la plateforpme de gérer le compte de chacun de ses employés
\end{reqf}

\begin{reqf}{Le système doit permettre au gestionnaire de la plateforme de récupérer le mot de passe des comptes}
\item Le système doit permettre au gestionnaire de la plateforme de récupérer le mot de passe de son compte par email.
\end{reqf}



%%% Fin Besoins utilisateurs GdP


%%% BESOINS UTILISATEURS

A REMETTRE DANS LE BON ORDRE ICI ET AUSSI DANS BESOINS UTILISATEURS !!!
+ VOIR POUR EN SÉPARER OU NON
AJOUTER EVENTUELLEMENT LA POSSIBILITÉ D'AVOIR ACCÈS À L'HISTORIQUE DE SES COMMANDES, PEUT ETRE PLUS JUDICIEUX QUE DE LE METTRE DANS PLEIN DE BESOINS DIFFÉRENTS ?

\begin{reqf}{Le système doit permettre à un utilisateur de s'inscrire}
\item L'utilisateur a accès à une page de création de compte sur la page d'accueil de la plateforme
\item La page de création de compte offre la possibilité de créer un compte "client"
\end{reqf}

\begin{reqf}{Le système doit permettre à un utilisateur de se désinscrire}
\item L'utilisateur peut accéder à ses paramètres personnels
\item Sur sa page personnel, l'utilisateur a accès à un bouton permettant de se désinscrire de la plateforme
\item Le système doit demander une confirmation de suppression
\end{reqf}

\begin{reqf}{Le système doit permettre à un utilisateur de pouvoir demander la création d'un nouveau mot de passe}
\item Depuis l'écran de connexion, l'utilisateur aura accès à une option "Mot de passe oublié".
\item Cette option lui demandera de saisir son adresse mail, et enverra le mot de passe temporaire à l'utilisateur.
\item Lors de la connexion d'un utilisateur avec un mot de passe temporaire, le système force l'utilisateur a changer de mot de passe.
\end{reqf}

\begin{reqf}{Le système doit permettre à un utilisateur de sélectionner un point de collecte}
\item Une fois connecté, l'utilisateur doit saisir une localisation par code postal ou nom de ville.
\item Le système permet une autocomplétion et la validation de ces données.
\item Le système doit afficher tous les points de collecte qui se trouvent dans un rayon de 20km autour de cette localisation
\item En fonction de la liste de points de collecte, l'utilisateur peut en sélectionner un en cliquant dessus.
\end{reqf}

\begin{reqf}{Le système doit permettre à un utilisateur de pouvoir commander des produits}
\item Le système affiche tous les produits disponibles à un point de collecte, ainsi que la date limite pour commander/annuler.
\item L'utilisateur peut choisir une quantité entière de produit (1,2,..., n unités).
\item L'utilisateur peut ajouter le nombre d'unités choisi dans son panier.
\end{reqf}

\begin{reqf}{Le système doit permettre à un utilisateur de pouvoir annuler une commande}
\item L'utilisateur a accès à toutes ses commandes passées dans son espace personnel
\item L'utilisateur doit pouvoir visualiser l'état des commandes passées.
\item Si le délais fixé par le responsable du point de collecte pour commander n'est pas dépassé, l'utlisateur doit pouvoir annuler sa commande à l'aide d'un bouton.
\item Le système doit envoyer une confirmation d'annulation de commande.
\item Le système doit envoyer une notification d'annulation automatiquement pas mail à l'utilisateur.
\item La commande annulée doit resté présente dans l'historique de commandes avec la mention de l'annulation.
\end{reqf}

\begin{reqf}{Le système doit permettre à un utilisateur de visualiser les informations concernant les produits}
\item L'utilisateur doit pouvoir facilement visualiser les prix.
\item L'utilisateur doit pouvoir facilement visualiser les quantités
\item L'utilisateur doit pouvoir facilement visualiser la provenance des produits.
\end{reqf}

\begin{reqf}{Le système doit permettre à un utilisateur de visualiser le contenu de son panier}
\item L'utilisateur doit avoir la possibilité à tout moment de consulter le contenu de son panier grâce à un bouton
\item Dans son panier, l'utilisateur visualise tous les produits qu'il a sélectionner, ainsi que les quantités, les prix et un prix total de commande.
\item L'utilisateur doit pouvoir modifier les quantités dans son panier sur les produits qu'il a sélectionné.
\item L'utilisateur doit pouvoir supprimer des éléments de son panier
\item L'utilisateur doit pouvoir valider son panier.
\end{reqf}

\begin{reqf}{Le système doit permettre à un utilisateur de payer sa commande en ligne}
\item Le système doit permettre la redirection de l'utilisateur vers un service de paiement en ligne pour payer sa commande.
\item Le système de paiement doit être sécurisé.
\end{reqf}

\begin{reqf}{Le système doit permettre à un utilisateur le remboursement de sa commande}
\item Lorsqu'un utilisateur a annulé sa commande, le système doit gérer automatiquement son remboursement une dois le délais de commande fixé par le responsable du point de collecte est dépassée.
\end{reqf}

\begin{reqf}{Le système doit permettre à un utilisateur de signaler un problème}
\item Dans son historique de commande, l'utilisateur doit pouvoir signaler un problème concernant une commande particulière via une mention "signaler un problème".
\item Pour faciliter le traitement, l'utilisateur doit avoir le choix entre plusieurs catégories de problèmes : problème de remboursement, problème de quantité non respectée, problème concernant la qualité d'un produit, problème concernant un point de collecte, problème concernant un point de collecte.
\end{reqf}


%%% FIN BESOIN UTILISATEURS



\section{Besoins système non fonctionnels}
$<$Certains besoins non fonctionnels peuvent nécessiter une description plus précise mais ce n'est pas obligatoire.$>$
Pas grand chose à ajouter
Performance peut-être détailler en fonction d'une partie du site
Cryptage des mots de passe (avec truc sur le client)


\section{Interfaces utilisateurs}
$<$Ici vous pouvez donner des éléments relatifs aux interactions avec l'utilisateur (écrans, appareils).$>$
Page avec les produits d'un PC et les coups de coeur
\newpage
\chapter{Appendices}
\section{Appendice A: Glossaire}
%see https://en.wikibooks.org/wiki/LaTeX/Glossary
\printglossaries

\section{Appendice B: Modèles d'analyse}
$<$Ajoutez ici les diagrammes et modèles qui peuvent servir à la compréhension du cahier des charges.$>$
2 DIAGRAMMES SEQUENCES QUENTIN
1 DIAGRAMME SEQUENCE SARAH
DIAGRAMME D'ETATS (VENTE) QUENTIN
3 max diagrammes de séquence (prendre l'exemple du paiement avec utilisateur, application et système de paiement)

\subsubsection{Modèle de données}
$<$Diagramme entité association ou diagramme de classe correspondant aux informations nécessaires à l'application.$>$

\begin{figure}
\includegraphics[width=\textwidth]{"AbsencesClasses"}
\caption{Exemple de diagramme de classe}
\end{figure}

\begin{figure}
\includegraphics[width=\textwidth]{"img/DiagrammeUML"}
\caption{Modèle de données}
\end{figure}


% \subsubsection{Dictionnaire de données}
% $<$Le dictionnaire de données est un tableau présentant pour chaque donnée son nom, son type et sa définition.$>$

% \begin{center}
% \begin{tabular}{ |l|l|l| } 
%  \hline
%  Nom & Type  & Description \\ 
%  \hline\hline
%  email & text & adresse de courrier électronique des utilisateurs \\ 
%  \hline
%  password & text & mot de passe pour l'authentification \\ 
%  \hline
% \end{tabular}
% \end{center}
\newpage
\bibliographystyle{plain}
\bibliography{biblio}
\end{document}