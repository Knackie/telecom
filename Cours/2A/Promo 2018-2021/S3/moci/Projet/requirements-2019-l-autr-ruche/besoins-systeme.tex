\chapter{Besoins système}

\section{Besoins système fonctionnels}
CHACUN REPREND CEUX DE SA PARTIE
$<$Pour chaque besoin fonctionnel, donnez une description détaillée de son fonctionnement permettant de comprendre plus précisément son fonctionnement. La numérotation doit correspondre aux besoins utilisateurs.$>$


\begin{reqf}{Le système doit permettre d'éditer la liste des absences pour un étudiant.}
\item L'utilisateur peut rechercher un étudiant à partir de son nom ou de son numéro INE. 
\item La recherche est facilitée par un mécanisme d'auto-complétion. 
\item La liste des absences s'affiche à l'écran : date, motif, justification.
\item Le total des absences justifiées et non justifiées est également affiché.
\end{reqf}

%%% BESOINS RESPONSABLE PC

\begin{reqf}{Le système doit permettre d'éditer la liste des absences pour un étudiant.}
\item L'utilisateur peut rechercher un étudiant à partir de son nom ou de son numéro INE. 
\item La recherche est facilitée par un mécanisme d'auto-complétion. 
\item La liste des absences s'affiche à l'écran : date, motif, justification.
\item Le total des absences justifiées et non justifiées est également affiché.
\end{reqf}

\begin{reqf}{Le système doit permettre à un responsable de point de collecte de s'inscrire au site.}
\item L'utilisateur a accès à une page de création de compte.
\end{reqf}

\begin{reqf}{Le système doit permettre à un responsable de point de collecte de proposer la création d'un nouveau point de collecte.}
\item Le responsable peut saisir les coordonnées du point de collecte qu'il veut créer.
\item Le responsable a accès à sa liste des points de collecte et un témoin lui indique son état (créé, en attente, création refusée).
\end{reqf}

\begin{reqf}{Le système doit permettre à un responsable de point de collecte d'accepter la participation d'un producteur dans un point de collecte dont il est responsable.}
\item En ouvrant la fiche d'un point de collecte, le responsable a accès à la liste des producteurs souhaitant faire parties du point de collecte.
\item Le clic sur un producteur ouvre une fenêtre lui permettant d'accepter la collaboration ou non.
\end{reqf}

\begin{reqf}{Le système doit permettre à un responsable de point de collecte de refuser la participation d'un producteur dans un point de collecte dont il est responsable.}
\item En ouvrant la fiche d'un point de collecte, le responsable a accès à la liste des producteurs souhaitant faire parties du point de collecte.
\item Le clic sur un producteur ouvre une fenêtre lui permettant d'accepter la collaboration ou non.
\end{reqf}

\begin{reqf}{Le système doit permettre à un responsable de point de collecte d'arrêter sa collaboration avec un producteur.}
\item En ouvrant la fiche d'un point de collecte, le responsable a accès à la liste des producteurs faisant parties du point de collecte.
\item Le clic sur un producteur ouvre une fenêtre lui permettant d'arrêter la collaboration ou non.
\end{reqf}

\begin{reqf}{Le système doit permettre à un responsable de point de collecte définir le jour et l'heure de remise des paniers.}
\item En ouvrant la fiche d'un point de collecte, le responsable a accès aux paramètres du point de collecte.
\item Dans les paramètres du point de collecte se trouvent le jour et la remise des paniers.
\end{reqf}

\begin{reqf}{Le système doit permettre à un responsable de point de collecte d'ouvrir les ventes sur un point de collecte.}
\item En ouvrant la fiche d'un point de collecte, le responsable peut ouvrir les ventes du point de collecte.
\end{reqf}

\begin{reqf}{Le système doit permettre à un responsable de point de collecte de fermer les ventes sur un point de collecte.}
\item En ouvrant la fiche d'un point de collecte, le responsable peut fermer les ventes du point de collecte.
\end{reqf}

\begin{reqf}{Le système doit permettre à un responsable de point de pointer les livraisons.}
\item Depuis la fiche d'un point de collecte, le responsable a accès à l'inventaire des produits commandés.
\item Cet inventaire lui permet de cocher la bonne réception des produits et d'indiquer les éventuelles remarques concernant ces derniers.
\end{reqf}

\begin{reqf}{Le système doit permettre à un responsable de point de collecte de reporter les manquements ou les erreurs.}
\item Depuis la fiche d'un point de collecte, le responsable a accès à l'inventaire des produits commandés.
\item Cet inventaire lui permet de cocher la bonne réception des produits et d'indiquer les éventuelles remarques concernant ces derniers.
\item Dans l'inventaire le responsable peut saisir les quantités manquantes ou en surplus.
\end{reqf}

\begin{reqf}{Le système doit permettre à un responsable de point de collecte de mettre en avant certains articles.}
\item Depuis la fiche d'un point de collecte, le responsable a accès à la liste de tous les articles mis en vente pendant le cycle.
\item Depuis la liste de tous les articles mis en vente pendant le cycle, le responsable peut à l'aide d'un bouton sélectionner des articles qui apparaîtront en tête de liste.
\end{reqf}

\begin{reqf}{Le système doit permettre à un responsable de point de collecte de définir les cycles de vente d'un point de collecte.}
\item Depuis la fiche d'un point de collecte, le responsable a accès aux paramètres du point de colecte.
\item Depuis les paramètres du point de collecte, le responsable peut définir le nombre de jours que dure un cycle de vente.
\end{reqf}

\begin{reqf}{Le système doit permettre à un responsable de point de collecte de demander la fermeture définitive d'un point de collecte.}
\item Depuis la fiche d'un point de collecte, le responsable a accès aux paramètres du point de colecte.
\item Depuis les paramètres du point de collecte, le responsable peut demander la fermeture définitive d'un point de collecte que le système lui demandera de confirmer.
\end{reqf}

\begin{reqf}{Le système doit permettre à un responsable de point de collecte de définir une date limite de commande.}
\item Depuis la fiche d'un point de collecte, le responsable a accès aux paramètres du point de colecte.
\item Depuis les paramètres du point de collecte, le responsable peut définir le nombre de jours avant lesquels il n'est plus possible de commander.
\end{reqf}

\begin{reqf}{Le système doit permettre à un responsable de point de collecte de se désinscrire du site.}
\item Depuis sa fiche utilisateur, le responsable peut demander sa désinscription au site, le système lui demandera une confirmation.
\item Si un responsable se désinscrit du site, tous les points de collecte dont il est responsable seront soit transférés à un autre responsable par le gestionnaire, soit fermé par le gestionnaire.
\end{reqf}

\begin{reqf}{Le système doit permettre à un responsable de point de collecte de fermer temporairement un point de collecte.}
\item Depuis la fiche d'un point de collecte, le responsable a accès aux paramètres du point de colecte.
\item Depuis les paramètres du point de collecte, le responsable peut demander la fermeture temporaire du point de collecte que le système lui demandera de confirmer.
\end{reqf}

\begin{reqf}{Le système doit permettre à un responsable de point de collecte de rouvrir un point de collecte qu'il a fermé.}
\item Depuis la fiche d'un point de collecte, le responsable a accès aux paramètres du point de colecte.
\item Depuis les paramètres du point de collecte, le responsable peut demander la ré-ouverture du point de collecte que le système lui demandera de confirmer.
\end{reqf}

\begin{reqf}{Le système doit permettre à un responsable de point de collecte de changer de mot de passe en cas d'oubli.}
\item Depuis l'écran de connexion, l'utilisateur aura accès à une option "Mot de passe oublié".
\item Cette option lui demandera de saisir son adresse email, et enverra le mot de passe temporaire à l'utilisateur.
\item Lors de la connexion d'un utilisateur avec un mot de passe temporaire, le système force l'utilisateur a changer de mot de passe.
\end{reqf}


%%% FIN BESOINS RESPONSABLE PC



%%% Besoins utilisateurs Gestionnaire de plateforme

\begin{reqf}{Le système doit permettre au gestionnaire de la plateforme de valider la création d'un point de collecte}
\item Le gestionnaire de la plateforme doit pouvoir voir les demandes de création de points de collecte
\item Le gestionnaire de la plateforme doit pouvoir valider les demandes de création de points de collecte
\end{reqf}

\begin{reqf}{Le système doit permettre au gestionnaire de la plateforme d'invalider la création d'un point de collecte}
\item Le gestionnaire de la plateforme doit pouvoir voir les demandes de création de points de collecte
\item Le gestionnaire de la plateforme doit pouvoir proposer des suggestions de modifications
\item Le gestionnaire de la plateforme doit pouvoir invalider les demandes de création de points de collecte
\end{reqf}

\begin{reqf}{Le système doit permettre au gestionnaire de la plateforme de fermer un point de collecte}
\item Le gestionnaire de la plateforme doit pouvoir fermer un point de collecte
\end{reqf}

\begin{reqf}{Le système doit permettre au gestionnaire de la plateforme de visualiser les informations d'un point de collecte}
\item Le gestionnaire de la plateforme doit pouvoir voir le chiffre d'affaire du point de collecte
\item Le gestionnaire de la plateforme doit pouvoir voir sa fréquentation du point de collecte
\item Le gestionnaire de la plateforme doit pouvoir voir la localisation du point de collecte
\item Le gestionnaire de la plateforme doit pouvoir voir le taux de remboursement du point de collecte
\item Le gestionnaire de la plateforme doit pouvoir voir le pourcentage du chiffre d'affaire du point de collecte destiné au responsable du point de collecte
\item Le gestionnaire de la plateforme doit pouvoir voir le pourcentage du chiffre d'affaire du point de collecte destiné au gestionnaire de la plateforme
\end{reqf}

\begin{reqf}{Le système doit permettre au gestionnaire de la plateforme de définir le pourcentage de CA que prend le responsable d'un point de collecte}
\item Le gestionnaire de la plateforme doit pouvoir voir le pourcentage du chiffre d'affaire du point de collecte destiné au responsable du point de collecte
\item Le gestionnaire de la plateforme doit pouvoir définir le pourcentage du chiffre d'affaire du point de collecte destiné au responsable du point de collecte
\end{reqf}

\begin{reqf}{Le système doit permettre au gestionnaire de la plateforme de définir le pourcentage de CA que prend le gestionnaire de la plateforme}
\item Le gestionnaire de la plateforme doit pouvoir voir le pourcentage du chiffre d'affaire du point de collecte destiné au gestionnaire de la plateforme
\item Le gestionnaire de la plateforme doit pouvoir définir le pourcentage du chiffre d'affaire du point de collecte destiné au gestionnaire de la plateforme
\end{reqf}

\begin{reqf}{Le système doit permettre au gestionnaire de la plateforme de rembourser les clients en fonction des erreurs remontées par responsable du point de collecte}
\item Le système doit afficher au gestionnaire du point de collecte les erreurs remontées par les responsables de point de collecte
\item Le système doit permettre de déclencher le remboursement pour les produits en question
\end{reqf}

\begin{reqf}{Le système doit permettre au gestionnaire de la plateforme de consulter les remontés des responsables des points de collecte}
\item Le système doit afficher au gestionnaire du point de collecte les erreurs remontées par les responsables de point de collecte
\end{reqf}

\begin{reqf}{Le système doit permettre au gestionnaire de la plateforme de contrôler un point de collecte}
\item Le système doit permettre au gestionnaire de la plateforme d'accéder aux informations sur d'un point de collecte
\item Le système doit permettre d'ajouter des notes sur un point de collecte lors d'un controle
\item Le système doit permettre de programmer un nouveau controle si besoin -
\end{reqf}

\begin{reqf}{Le système doit permettre au gestionnaire de la plateforme de créer des comptes pour les employés du gestionnaire de la plateforme}
\item Le système doit permettre au gestionnaire de la plateforpme de créer un compte pour chacun de ses employés
\item Le système doit permettre au gestionnaire de la plateforpme de gérer le compte de chacun de ses employés
\end{reqf}

\begin{reqf}{Le système doit permettre au gestionnaire de la plateforme de récupérer le mot de passe des comptes}
\item Le système doit permettre au gestionnaire de la plateforme de récupérer le mot de passe de son compte par email.
\end{reqf}



%%% Fin Besoins utilisateurs GdP


%%% BESOINS UTILISATEURS

A REMETTRE DANS LE BON ORDRE ICI ET AUSSI DANS BESOINS UTILISATEURS !!!
+ VOIR POUR EN SÉPARER OU NON
AJOUTER EVENTUELLEMENT LA POSSIBILITÉ D'AVOIR ACCÈS À L'HISTORIQUE DE SES COMMANDES, PEUT ETRE PLUS JUDICIEUX QUE DE LE METTRE DANS PLEIN DE BESOINS DIFFÉRENTS ?

\begin{reqf}{Le système doit permettre à un utilisateur de s'inscrire}
\item L'utilisateur a accès à une page de création de compte sur la page d'accueil de la plateforme
\item La page de création de compte offre la possibilité de créer un compte "client"
\end{reqf}

\begin{reqf}{Le système doit permettre à un utilisateur de se désinscrire}
\item L'utilisateur peut accéder à ses paramètres personnels
\item Sur sa page personnel, l'utilisateur a accès à un bouton permettant de se désinscrire de la plateforme
\item Le système doit demander une confirmation de suppression
\end{reqf}

\begin{reqf}{Le système doit permettre à un utilisateur de pouvoir demander la création d'un nouveau mot de passe}
\item Depuis l'écran de connexion, l'utilisateur aura accès à une option "Mot de passe oublié".
\item Cette option lui demandera de saisir son adresse mail, et enverra le mot de passe temporaire à l'utilisateur.
\item Lors de la connexion d'un utilisateur avec un mot de passe temporaire, le système force l'utilisateur a changer de mot de passe.
\end{reqf}

\begin{reqf}{Le système doit permettre à un utilisateur de sélectionner un point de collecte}
\item Une fois connecté, l'utilisateur doit saisir une localisation par code postal ou nom de ville.
\item Le système permet une autocomplétion et la validation de ces données.
\item Le système doit afficher tous les points de collecte qui se trouvent dans un rayon de 20km autour de cette localisation
\item En fonction de la liste de points de collecte, l'utilisateur peut en sélectionner un en cliquant dessus.
\end{reqf}

\begin{reqf}{Le système doit permettre à un utilisateur de pouvoir commander des produits}
\item Le système affiche tous les produits disponibles à un point de collecte, ainsi que la date limite pour commander/annuler.
\item L'utilisateur peut choisir une quantité entière de produit (1,2,..., n unités).
\item L'utilisateur peut ajouter le nombre d'unités choisi dans son panier.
\end{reqf}

\begin{reqf}{Le système doit permettre à un utilisateur de pouvoir annuler une commande}
\item L'utilisateur a accès à toutes ses commandes passées dans son espace personnel
\item L'utilisateur doit pouvoir visualiser l'état des commandes passées.
\item Si le délais fixé par le responsable du point de collecte pour commander n'est pas dépassé, l'utlisateur doit pouvoir annuler sa commande à l'aide d'un bouton.
\item Le système doit envoyer une confirmation d'annulation de commande.
\item Le système doit envoyer une notification d'annulation automatiquement pas mail à l'utilisateur.
\item La commande annulée doit resté présente dans l'historique de commandes avec la mention de l'annulation.
\end{reqf}

\begin{reqf}{Le système doit permettre à un utilisateur de visualiser les informations concernant les produits}
\item L'utilisateur doit pouvoir facilement visualiser les prix.
\item L'utilisateur doit pouvoir facilement visualiser les quantités
\item L'utilisateur doit pouvoir facilement visualiser la provenance des produits.
\end{reqf}

\begin{reqf}{Le système doit permettre à un utilisateur de visualiser le contenu de son panier}
\item L'utilisateur doit avoir la possibilité à tout moment de consulter le contenu de son panier grâce à un bouton
\item Dans son panier, l'utilisateur visualise tous les produits qu'il a sélectionner, ainsi que les quantités, les prix et un prix total de commande.
\item L'utilisateur doit pouvoir modifier les quantités dans son panier sur les produits qu'il a sélectionné.
\item L'utilisateur doit pouvoir supprimer des éléments de son panier
\item L'utilisateur doit pouvoir valider son panier.
\end{reqf}

\begin{reqf}{Le système doit permettre à un utilisateur de payer sa commande en ligne}
\item Le système doit permettre la redirection de l'utilisateur vers un service de paiement en ligne pour payer sa commande.
\item Le système de paiement doit être sécurisé.
\end{reqf}

\begin{reqf}{Le système doit permettre à un utilisateur le remboursement de sa commande}
\item Lorsqu'un utilisateur a annulé sa commande, le système doit gérer automatiquement son remboursement une dois le délais de commande fixé par le responsable du point de collecte est dépassée.
\end{reqf}

\begin{reqf}{Le système doit permettre à un utilisateur de signaler un problème}
\item Dans son historique de commande, l'utilisateur doit pouvoir signaler un problème concernant une commande particulière via une mention "signaler un problème".
\item Pour faciliter le traitement, l'utilisateur doit avoir le choix entre plusieurs catégories de problèmes : problème de remboursement, problème de quantité non respectée, problème concernant la qualité d'un produit, problème concernant un point de collecte, problème concernant un point de collecte.
\end{reqf}


%%% FIN BESOIN UTILISATEURS



\section{Besoins système non fonctionnels}
$<$Certains besoins non fonctionnels peuvent nécessiter une description plus précise mais ce n'est pas obligatoire.$>$
Pas grand chose à ajouter
Performance peut-être détailler en fonction d'une partie du site
Cryptage des mots de passe (avec truc sur le client)


\section{Interfaces utilisateurs}
$<$Ici vous pouvez donner des éléments relatifs aux interactions avec l'utilisateur (écrans, appareils).$>$
Page avec les produits d'un PC et les coups de coeur